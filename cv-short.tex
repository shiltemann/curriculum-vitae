%!TEX TS-program = xelatex
\documentclass[]{shiltemann-cv}
%\addbibresource{publications.bib}
\usepackage{fontspec}
\defaultfontfeatures{Path = /usr/share/texlive/texmf-dist/fonts/opentype/public/fontawesome/}
\usepackage{fontawesome}

\begin{document}
\header{Saskia}{Hiltemann}
    {Post-Doctoral Researcher, Bioinformatics \& Education}


% In the aside, each new line forces a line break
\begin{aside}
  \includegraphics[width=75pt]{photo-saskia.png}
  \section{About}
    Saskia Hiltemann, Ph.D.
    ~
    Thérèse Schwartzestraat 24
    2597 XJ, Den Haag
    The Netherlands
    ~
    saskiahilltemann @gmail.com \faEnvelope
    shiltemann \faGithub \ \faTwitter \ \faLinkedin
    \ \\
    ORCiD:
    0000-0003-3803-468X \faOrcid
  \section{Languages}
    Fluent: Dutch,English
    Some: German \& French
  \section{Tech Skills}
    Galaxy, UNIX, GitHub, Jekyll, Python, C, C++, Docker, LaTeX, HTML, CSS, Javascript
  \section{Other Skills}
    Training, Community Building, Outreach \& Dissemination
\end{aside}

\section{Interests}

Bioinformatics, Education \& Training, Open Science, FAIR principles, Community Building, Genome Sequencing, Cancer Research, Microbiome Analysis, Workflows, Automation, Visualisation, Best Practices, Infosec, CTF, Traveling, Hiking, Reading, Puzzles.


\section{Projects}
\emph{Selection of recent projects.}

\begin{entrylist}
    \entry
    {2015-current}
    {Galaxy Training Network (GTN): Training Materials, Infrastructure \& Community Building}
    {GalaxyProject}
    {Co-founder of the GTN infrastructure for the collaborative development of Galaxy training materials. Community building efforts to ensure utility for educators.  Development of numerous tutorials on the topics of Metagenomics, Sequence Analysis, Galaxy Development and Visualisation. https://training.galaxyproject.org }

\end{entrylist}

\begin{entrylist}

   \entry
    {2020-2023}
    {Gallantries Project, Project Coordinator \& Work Package Lead}
    {ErasmusMC, Erasmus+ KA203 grant}
    {Development of bioinformatics curricula for higher education and later-career reserachers training workshops. Building of training infrastructure for educators, with a focus on remote training}

\end{entrylist}
\begin{entrylist}

   \entry
    {2019-2023}
    {CINECA project, Work Package co-lead of Training \& Dissemination}
    {ErasmusMC, Horizon2020 grant}
    {CINECA (Common Infrastructure for National Cohorts in Europe, Canada, and Africa) aims at creating a global infrastructure for fedarated data analysis. Co-lead of the Training \& Dissemination  workpackage, and also involved in the Healthcare Interoperability \& Clinical Applications work package.}
 \end{entrylist}



\section{Publications}
\emph{Selection of scientific publications.}

\begin{entrylist}
\entry
  {2023}
  {Galaxy Training: A powerful framework for teaching!}
  {PLOS Computational Biology}
  {DOI: 10.1371/journal.pcbi.1010752 }

\end{entrylist}

\begin{entrylist}
\entry
 {2018}
 {Community-driven data analysis training for biology}
 {Cell Systems}
 {DOI: 10.1016/j.cels.2018.05.012}

\end{entrylist}
\begin{entrylist}

\entry
  {2018}
  {Development and evaluation of a culture-free microbiota profiling platform (MYcrobiota) for clinical diagnostics}
  {European Journal of Clinical Microbiology \& Infectious Diseases}
  {DOI: 10.1007/s10096-018-3220-z}


\end{entrylist}
\begin{entrylist}

\entry
  {2015}
  {Discriminating somatic and germline mutations in tumor DNA samples without matching normals }
  {Genome Research}
  {DOI: 10.1101/gr.183053.114}

\end{entrylist}

For a full list of publications, please see ORCiD: \url{https://orcid.org/0000-0003-3803-468X}

\section{Community Activities}
\begin{entrylist}
\entry
  {ongoing}
  {MicroGalaxy Community co-lead}
  {Community Building, Tool development, workflow development, training }
  {Creating a community around microbial analysis in Galaxy. Identifying needs of the community and developing solutions to address those needs.}

\end{entrylist}
\begin{entrylist}

\entry
  {ongoing}
  {Galaxy Intergalactic Utilities Commision (IUC) member}
  {Galaxy Tool development}
  {Best-practice Galaxy tool development and maintenance.}
  {}
\end{entrylist}
\begin{entrylist}

\entry
  {ongoing}
  {Galaxy Working Groups}
  {Galaxy, Training, Outreach}
  {Member of the Galaxy Outreach \& Training working group. }
  {}

\end{entrylist}

\section{Employment History}

\begin{entrylist}
 \entry
    {2023-current}
    {Albert-Ludwigs-Universität Freiburg}
    {Post-doctoral Bioinformatician}
    {Scientific tool developer and data steward for the MAdLand (\emph{Molecular Adaptation to Land: plant evolution to change}) project.}
 \entry
    {2012-2023}
    {Erasmus University Medical Center, Rotterdam}
    {(Post-) Doctoral Bioinformatician \& Scientific Researcher}
    {Galaxy tool and workflow development, Education \& Curriculum Development, Microbiota Analysis and Antibiotic Resistance Detection, Prostate Cancer Analysis, Software Development and Pipeline building, Galaxy System's Administrator, Training, Outreach \& Dissemination, Federated Data Analysis Infrastructure Development, FAIR data analysis}
  \entry
    {2010-2012}
    {After's Cool, The Hague}
    {Science tutoring}
    {Tutor of High School Students. \emph{Math, Physics, Chemistry}}
\end{entrylist}


\section{Education}

\begin{entrylist}
  \entry
    {2021}
    {Ph.D. Bioinformatics}
    {Erasmus Medical Center}
    {\emph{Thesis:} "Jigsaw Genomics: Assembling the pieces toward open and accessible bioinformatics for everyone"}
\end{entrylist}
\begin{entrylist}


  \entry
    {2010}
    {M.Sc.}
    {LIACS, University of Leiden \& TU Delft}
    {Computer Science, Specialization in Bioinformatics}

\end{entrylist}
\begin{entrylist}

  \entry
    {2008}
    {B.Sc.}
    {LIACS, University of Leiden}
    {Computer Science}
\end{entrylist}



\end{document}
